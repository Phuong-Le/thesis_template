\chapter{Writing styles}

I asked my brilliant friend Soraya Zwahlen for her advice regarding writing a thesis, here is her response:

``\begin{itemize}
    \item \textbf{Have a story:}
Start off by dot-pointing your ideas. Then, organise them in an order that tells a logical story. This doesn't necessarily have to be in chronological order - start by describing experiments that 'lay the groundwork', i.e. establish the basics which later experiments build on.
    \item \textbf{Have a structure:}
After dot-pointing your ideas in a logical order, start working on the individual paragraphs. Each paragraph should start with a topic sentence that summarises the main point of the paragraph. The best way to check whether your paragraphs are in a logical order and flow is to go back and read just the FIRST sentence of each paragraph, and see if it makes sense.
    \item \textbf{Write to engage, not to impress:}
Use simple language. Avoid acronyms and complicated words. Even if you're writing for a scientific journal, assume the reader is not familiar with your research. You want as many people as possible to read your science - make it easy for them!
    \item \textbf{Edit, edit, edit:}
Leave enough time to edit your text extensively! Only include what's relevant to the story - don't be afraid of cutting whole sentences or even paragraphs! 
\end{itemize}''

Soraya added that \textbf{``Write to engage, not to impress''} was the most important point. I take that as follows: your goal is not to impress people but to make them understand you, because they can only be impressed if they understand...

I would like to add some of my points

\begin{itemize}
    \item \textbf{Consider who your readers are:} Although you are meant to write for the general audience, it is worth considering who are examining your thesis, thereby contributing to your long term academic career. Are they an expert? In which case they might want detailed explanation of your techniques. Are they an absolute outsider? In which case you should be super simple, and put extra focus on the abstract and the motivation behind your work. Are they in a slightly similar field as you, but not really, such that they have assumptions prior to reading your work? In which case, you want to try to identify any potential misunderstandings. This is tricky, but one way to do it is to talk with them directly.  
    \item \textbf{Try to use graphics (\textit{e.g.} diagrams) to explain complicated ideas:} I have met a lot of people who tend to skim all the texts and just focus at the tables and figures and their captions. This can be annoying, but what can you do about it? Well, use more figures or diagrams, particularly as you describe your methods! But remember, diagrams are to simplify, not to complicate things up. It is good to have very descriptive captions to accompany the figures as well. Making diagrams can be tedious, but it's good in that it helps with your own understanding as well. You've got this!
    \item \textbf{Good luck and I hope you enjoy writing!}
\end{itemize}